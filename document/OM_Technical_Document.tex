\documentclass[]{article}
\usepackage{lmodern}
\usepackage{amssymb,amsmath}
\usepackage{ifxetex,ifluatex}
\usepackage{fixltx2e} % provides \textsubscript
\ifnum 0\ifxetex 1\fi\ifluatex 1\fi=0 % if pdftex
  \usepackage[T1]{fontenc}
  \usepackage[utf8]{inputenc}
\else % if luatex or xelatex
  \ifxetex
    \usepackage{mathspec}
  \else
    \usepackage{fontspec}
  \fi
  \defaultfontfeatures{Ligatures=TeX,Scale=MatchLowercase}
\fi
% use upquote if available, for straight quotes in verbatim environments
\IfFileExists{upquote.sty}{\usepackage{upquote}}{}
% use microtype if available
\IfFileExists{microtype.sty}{%
\usepackage{microtype}
\UseMicrotypeSet[protrusion]{basicmath} % disable protrusion for tt fonts
}{}
\usepackage[margin=1in]{geometry}
\usepackage{hyperref}
\hypersetup{unicode=true,
            pdftitle={Phuong's Working Document},
            pdfborder={0 0 0},
            breaklinks=true}
\urlstyle{same}  % don't use monospace font for urls
\usepackage{color}
\usepackage{fancyvrb}
\newcommand{\VerbBar}{|}
\newcommand{\VERB}{\Verb[commandchars=\\\{\}]}
\DefineVerbatimEnvironment{Highlighting}{Verbatim}{commandchars=\\\{\}}
% Add ',fontsize=\small' for more characters per line
\usepackage{framed}
\definecolor{shadecolor}{RGB}{248,248,248}
\newenvironment{Shaded}{\begin{snugshade}}{\end{snugshade}}
\newcommand{\KeywordTok}[1]{\textcolor[rgb]{0.13,0.29,0.53}{\textbf{#1}}}
\newcommand{\DataTypeTok}[1]{\textcolor[rgb]{0.13,0.29,0.53}{#1}}
\newcommand{\DecValTok}[1]{\textcolor[rgb]{0.00,0.00,0.81}{#1}}
\newcommand{\BaseNTok}[1]{\textcolor[rgb]{0.00,0.00,0.81}{#1}}
\newcommand{\FloatTok}[1]{\textcolor[rgb]{0.00,0.00,0.81}{#1}}
\newcommand{\ConstantTok}[1]{\textcolor[rgb]{0.00,0.00,0.00}{#1}}
\newcommand{\CharTok}[1]{\textcolor[rgb]{0.31,0.60,0.02}{#1}}
\newcommand{\SpecialCharTok}[1]{\textcolor[rgb]{0.00,0.00,0.00}{#1}}
\newcommand{\StringTok}[1]{\textcolor[rgb]{0.31,0.60,0.02}{#1}}
\newcommand{\VerbatimStringTok}[1]{\textcolor[rgb]{0.31,0.60,0.02}{#1}}
\newcommand{\SpecialStringTok}[1]{\textcolor[rgb]{0.31,0.60,0.02}{#1}}
\newcommand{\ImportTok}[1]{#1}
\newcommand{\CommentTok}[1]{\textcolor[rgb]{0.56,0.35,0.01}{\textit{#1}}}
\newcommand{\DocumentationTok}[1]{\textcolor[rgb]{0.56,0.35,0.01}{\textbf{\textit{#1}}}}
\newcommand{\AnnotationTok}[1]{\textcolor[rgb]{0.56,0.35,0.01}{\textbf{\textit{#1}}}}
\newcommand{\CommentVarTok}[1]{\textcolor[rgb]{0.56,0.35,0.01}{\textbf{\textit{#1}}}}
\newcommand{\OtherTok}[1]{\textcolor[rgb]{0.56,0.35,0.01}{#1}}
\newcommand{\FunctionTok}[1]{\textcolor[rgb]{0.00,0.00,0.00}{#1}}
\newcommand{\VariableTok}[1]{\textcolor[rgb]{0.00,0.00,0.00}{#1}}
\newcommand{\ControlFlowTok}[1]{\textcolor[rgb]{0.13,0.29,0.53}{\textbf{#1}}}
\newcommand{\OperatorTok}[1]{\textcolor[rgb]{0.81,0.36,0.00}{\textbf{#1}}}
\newcommand{\BuiltInTok}[1]{#1}
\newcommand{\ExtensionTok}[1]{#1}
\newcommand{\PreprocessorTok}[1]{\textcolor[rgb]{0.56,0.35,0.01}{\textit{#1}}}
\newcommand{\AttributeTok}[1]{\textcolor[rgb]{0.77,0.63,0.00}{#1}}
\newcommand{\RegionMarkerTok}[1]{#1}
\newcommand{\InformationTok}[1]{\textcolor[rgb]{0.56,0.35,0.01}{\textbf{\textit{#1}}}}
\newcommand{\WarningTok}[1]{\textcolor[rgb]{0.56,0.35,0.01}{\textbf{\textit{#1}}}}
\newcommand{\AlertTok}[1]{\textcolor[rgb]{0.94,0.16,0.16}{#1}}
\newcommand{\ErrorTok}[1]{\textcolor[rgb]{0.64,0.00,0.00}{\textbf{#1}}}
\newcommand{\NormalTok}[1]{#1}
\usepackage{graphicx,grffile}
\makeatletter
\def\maxwidth{\ifdim\Gin@nat@width>\linewidth\linewidth\else\Gin@nat@width\fi}
\def\maxheight{\ifdim\Gin@nat@height>\textheight\textheight\else\Gin@nat@height\fi}
\makeatother
% Scale images if necessary, so that they will not overflow the page
% margins by default, and it is still possible to overwrite the defaults
% using explicit options in \includegraphics[width, height, ...]{}
\setkeys{Gin}{width=\maxwidth,height=\maxheight,keepaspectratio}
\IfFileExists{parskip.sty}{%
\usepackage{parskip}
}{% else
\setlength{\parindent}{0pt}
\setlength{\parskip}{6pt plus 2pt minus 1pt}
}
\setlength{\emergencystretch}{3em}  % prevent overfull lines
\providecommand{\tightlist}{%
  \setlength{\itemsep}{0pt}\setlength{\parskip}{0pt}}
\setcounter{secnumdepth}{0}
% Redefines (sub)paragraphs to behave more like sections
\ifx\paragraph\undefined\else
\let\oldparagraph\paragraph
\renewcommand{\paragraph}[1]{\oldparagraph{#1}\mbox{}}
\fi
\ifx\subparagraph\undefined\else
\let\oldsubparagraph\subparagraph
\renewcommand{\subparagraph}[1]{\oldsubparagraph{#1}\mbox{}}
\fi

%%% Use protect on footnotes to avoid problems with footnotes in titles
\let\rmarkdownfootnote\footnote%
\def\footnote{\protect\rmarkdownfootnote}

%%% Change title format to be more compact
\usepackage{titling}

% Create subtitle command for use in maketitle
\providecommand{\subtitle}[1]{
  \posttitle{
    \begin{center}\large#1\end{center}
    }
}

\setlength{\droptitle}{-2em}

  \title{Phuong's Working Document}
    \pretitle{\vspace{\droptitle}\centering\huge}
  \posttitle{\par}
    \author{}
    \preauthor{}\postauthor{}
    \date{}
    \predate{}\postdate{}
  

\begin{document}
\maketitle

{
\setcounter{tocdepth}{2}
\tableofcontents
}
\section{Description of the document}\label{description-of-the-document}

\begin{quote}
This is the Opportunity Mapping 2.0 Technical Document produced by the
Phuong Tseng. The intention is to capture changes and developments in
the 2019 version.
\end{quote}

\section{The Methodology Document}\label{the-methodology-document}

OM\_methodology\_v4\_Nov30.pdf

\subsection{Set-up}\label{set-up}

\subsection{A. The Domains}\label{a.-the-domains}

In 2019, there are 5 domains: education, economic \& mobility, housing
and neighborhood, conduit, and social capital. The social capital domain
is a new domain in 2019.

\subsubsection{1. Education Opportunity
Indicators}\label{education-opportunity-indicators}

This year, the education domain added a new indicator called Early
Childhood Participation Rate or Pre-K. Another indicator, adult with
bachelor's degree was moved from the education domain to the economic \&
mobility domain in 2019.

\begin{Shaded}
\begin{Highlighting}[]
\NormalTok{edu_list <-}
\StringTok{  }\KeywordTok{c}\NormalTok{(}
  \StringTok{"fips"}\NormalTok{,}
  \StringTok{"CountyID.x"}\NormalTok{,}
  \StringTok{"TOTPOP.x"}\NormalTok{,}
  \StringTok{"math_prof"}\NormalTok{,}
  \StringTok{"read_prof"}\NormalTok{,}
  \StringTok{"grad_rate"}\NormalTok{,}
  \StringTok{"pct_not_frpm"}\NormalTok{,}
  \StringTok{"z_math_prof"}\NormalTok{,}
  \StringTok{"z_read_prof"}\NormalTok{,}
  \StringTok{"z_grad_rate"}\NormalTok{,}
  \StringTok{"az_pct_not_frpm"}\NormalTok{,}
  \StringTok{"county_name.x"}\NormalTok{,}
  \StringTok{"HD01_VD04"}\NormalTok{,}
  \StringTok{"HD01_VD03"}\NormalTok{,}
  \StringTok{"ratio"}\NormalTok{,}
  \StringTok{"ratio2"}\NormalTok{,}
  \StringTok{"z_preK"}
\NormalTok{  )}
\end{Highlighting}
\end{Shaded}

\subsubsection{2. Economic \& Mobility Opportunity
Indicators}\label{economic-mobility-opportunity-indicators}

There are a few changes to this domain in 2019. The adult with
bachelor's degree was added to this domain, median household income, and
median household value. Other indicators such as the commuting time and
entry-level jobs' measures were changed to TCAC's measures. A new
indicator, school district revenue per capita, was added to capture the
extent of municipal hoarding. Due to reliability issues of municipal
data, school district boundary was used as a proxy instead.

\begin{Shaded}
\begin{Highlighting}[]
\NormalTok{econ_list <-}\StringTok{ }\KeywordTok{c}\NormalTok{(}
  \StringTok{"fips"}\NormalTok{,}
  \StringTok{"CountyID.x"}\NormalTok{,}
  \StringTok{"TOTPOP.x"}\NormalTok{,}
  \StringTok{"total_pop_2017"}\NormalTok{,}
  \StringTok{"below_200_pov_2017.x"}\NormalTok{,}
  \StringTok{"moe_below_200_pov_2017.x"}\NormalTok{,}
  \StringTok{"pct_below_pov_2017"}\NormalTok{,}
  \StringTok{"moe_pct_below_pov_2017"}\NormalTok{,}
  \StringTok{"pct_below_200_pov_2017.x"}\NormalTok{,}
  \StringTok{"pct_assist_2017"}\NormalTok{,}
  \StringTok{"med_hhincome_2017"}\NormalTok{ ,}
  \StringTok{"moe_med_hhincome_2017"}\NormalTok{ ,}
  \StringTok{"employed_pop_20to60_2017"}\NormalTok{,}
  \StringTok{"pct_employed_20to60_2017"}\NormalTok{,}
  \StringTok{"home_value_2017"}\NormalTok{ ,}
  \StringTok{"moe_home_value_2017"}\NormalTok{,}
  \StringTok{"pct_bachelors_plus_2017"}\NormalTok{,}
  \StringTok{"above_200_pov_2017"}\NormalTok{,}
  \StringTok{"pct_above_200_pov_2017"}\NormalTok{,}
  \StringTok{"tot_hh_2017"}\NormalTok{,}
  \StringTok{"moe_tot_hh_2017"}\NormalTok{,}
  \StringTok{"moe_pct_long_commute_2017"}\NormalTok{,}
  \StringTok{"moe_assist_2017"}\NormalTok{,}
  \StringTok{"moe_long_commute_pct"}\NormalTok{,}
  \StringTok{"long_commute_pct"}\NormalTok{,}
  \StringTok{"low_wage_med_distance"}\NormalTok{ ,}
  \StringTok{"jobs_lowed"}\NormalTok{ ,}
  \StringTok{"rural_flag"}\NormalTok{,}
  \StringTok{"az_pct_assist_2017"}\NormalTok{ ,}
  \StringTok{"az_pct_employed_20to60_2017"}\NormalTok{,}
  \StringTok{"z_home_value_2017"}\NormalTok{ ,}
  \StringTok{"z_pct_bachelors_plus_2017"}\NormalTok{ ,}
  \StringTok{"az_pct_long_commute_2017"}\NormalTok{,}
  \StringTok{"z_jobs_lowed"}\NormalTok{ ,}
  \StringTok{"Econ_Domain"}\NormalTok{,}
  \StringTok{"z_sdrevpcap"}\NormalTok{,}
  \StringTok{"sdrev"}\NormalTok{,}
  \StringTok{"sdrevpcap"}\NormalTok{,}
  \StringTok{"sd_totpop"}
\NormalTok{  )}
\end{Highlighting}
\end{Shaded}

\subsubsection{3. Housing \& Neighborhood Opportunity
Indicators}\label{housing-neighborhood-opportunity-indicators}

The housing and neighborhood opportunity domain has two new
environmental indicators pulled from CalEnviroScreen (i.e.~pm25, lead).

\begin{Shaded}
\begin{Highlighting}[]
\NormalTok{housing_list <-}
\StringTok{  }\KeywordTok{c}\NormalTok{(}
  \StringTok{"fips"}\NormalTok{,}
  \StringTok{"CountyID.x"}\NormalTok{ ,}
  \StringTok{"TOTPOP.x"}\NormalTok{,}
  \StringTok{"county_name.x"}\NormalTok{,}
  \StringTok{"below_200_pov_2017.y"}\NormalTok{,}
  \StringTok{"moe_below_200_pov_2017.y"}\NormalTok{,}
  \StringTok{"pct_below_200_pov_2017.y"}\NormalTok{,}
  \StringTok{"pm25"}\NormalTok{,}
  \StringTok{"pct_pm25"}\NormalTok{,}
  \StringTok{"toxRelease"}\NormalTok{,}
  \StringTok{"pct_toxRelease"}\NormalTok{,}
  \StringTok{"lead_pctl"}\NormalTok{,}
  \StringTok{"pct_lead_pctl"}\NormalTok{ ,}
  \StringTok{"Grocery"}\NormalTok{,}
  \StringTok{"z_Grocery"}\NormalTok{ ,}
  \StringTok{"az_Grocery"}\NormalTok{,}
  \StringTok{"P_INSURED"}\NormalTok{ ,}
  \StringTok{"az_insurance"}\NormalTok{ ,}
  \StringTok{"H_Crime"}\NormalTok{,}
  \StringTok{"pct_parks"}\NormalTok{,}
  \StringTok{"az_pct_below_200_pov_2017"}\NormalTok{,}
  \StringTok{"az_pct_below_200_pov_20172"}\NormalTok{,}
  \StringTok{"az_pct_pm25"}\NormalTok{,}
  \StringTok{"az_pct_toxRelease"}\NormalTok{,}
  \StringTok{"az_pct_lead_pctl"}\NormalTok{ ,}
  \StringTok{"Housing_Env_Domain"}\NormalTok{,}
  \StringTok{"test_azcrime"}\NormalTok{ ,}
  \StringTok{"azhealthcare"}\NormalTok{ ,}
  \StringTok{"zparks"}
\NormalTok{  )}
\end{Highlighting}
\end{Shaded}

\subsubsection{4. Conduit}\label{conduit}

The Conduit domain has two indicators: median broadband download speed
and percentage of single-parent households.

\begin{Shaded}
\begin{Highlighting}[]
\NormalTok{conduit_list <-}
\StringTok{  }\KeywordTok{c}\NormalTok{(}
  \StringTok{"fips"}\NormalTok{,}
  \StringTok{"CountyID.x"}\NormalTok{,}
  \StringTok{"TOTPOP.x"}\NormalTok{,}
  \StringTok{"pct_singleparent_hh_2017.y"}\NormalTok{,}
  \StringTok{"moe_pct_singleparent_hh_2017.y"}\NormalTok{,}
  \StringTok{"az_pct_singleparent_hh_2017"}\NormalTok{,}
  \StringTok{"TOTPOP.y"}\NormalTok{,}
  \StringTok{"Median_bb"}\NormalTok{,}
  \StringTok{"z_broadband"}\NormalTok{,}
  \StringTok{"z_broadband2"}\NormalTok{,}
  \StringTok{"Conduit"}
\NormalTok{  )}
\end{Highlighting}
\end{Shaded}

\subsubsection{5. Social Capital}\label{social-capital}

This is our newest domain, which has the average distance to a religious
institution, registered voters voting rate, and average distance to club
membership and etc.

\begin{Shaded}
\begin{Highlighting}[]
\NormalTok{socap_list <-}
\StringTok{  }\KeywordTok{c}\NormalTok{(}
  \StringTok{"fips"}\NormalTok{,}
  \StringTok{"CountyID.x"}\NormalTok{,}
  \StringTok{"TOTPOP.x"}\NormalTok{,}
  \StringTok{"pct_singleparent_hh_2017.y"}\NormalTok{,}
  \StringTok{"moe_pct_singleparent_hh_2017.y"}\NormalTok{,}
  \StringTok{"az_pct_singleparent_hh_2017"}\NormalTok{,}
  \StringTok{"Clubs"}\NormalTok{,}
  \StringTok{"AVGDIS_REL"}\NormalTok{,}
  \StringTok{"reg_vote"}\NormalTok{,}
  \StringTok{"SOCIAL_CAP"}\NormalTok{,}
  \StringTok{"z_regvoter"}\NormalTok{,}
  \StringTok{"zreligious"}\NormalTok{,}
  \StringTok{"zclubs"}
\NormalTok{  )}
\end{Highlighting}
\end{Shaded}

\subsubsection{6. Compile All Indicators
Function}\label{compile-all-indicators-function}

\subsubsection{7. Calculate Domains}\label{calculate-domains}

\subsection{B. Index Calculation}\label{b.-index-calculation}

\begin{Shaded}
\begin{Highlighting}[]
\NormalTok{data}\OperatorTok{$}\NormalTok{index <-}\StringTok{ }\NormalTok{(data}\OperatorTok{$}\NormalTok{housing_domain }\OperatorTok{+}\StringTok{ }\NormalTok{data}\OperatorTok{$}\NormalTok{edu_domain }\OperatorTok{+}\StringTok{ }\NormalTok{data}\OperatorTok{$}\NormalTok{econ_domain }\OperatorTok{+}\StringTok{ }\NormalTok{data}\OperatorTok{$}\NormalTok{Socap_domain }\OperatorTok{+}\StringTok{ }\NormalTok{data}\OperatorTok{$}\NormalTok{Conduit_domain)}\OperatorTok{/}\DecValTok{5}

\CommentTok{#df <- data %>% select(fips,housing_domain,edu_domain,econ_domain,Socap_domain, Conduit_domain, index) %>% filter(is.na(index))}
\end{Highlighting}
\end{Shaded}

\subsection{C. Filters}\label{c.-filters}

\subsubsection{1. Filtering Single parent families \textgreater{}=
30\%}\label{filtering-single-parent-families-30}

returns 471 records with 8 NAs

\begin{Shaded}
\begin{Highlighting}[]
\CommentTok{#filter_function <- function(data, variable1, variabl2, value, value2)\{}
\CommentTok{#  data$variable1[which(data$variable2)] <- value2}
\CommentTok{#  return(data$variable)}
\CommentTok{#\}}

\NormalTok{data}\OperatorTok{$}\NormalTok{SPF_GT_}\DecValTok{30}\NormalTok{[}\KeywordTok{which}\NormalTok{(data}\OperatorTok{$}\NormalTok{pct_singleparent_hh_}\FloatTok{2017.}\NormalTok{y}\OperatorTok{<}\FloatTok{0.3}\NormalTok{)] <-}\StringTok{ }\DecValTok{0}

\NormalTok{data}\OperatorTok{$}\NormalTok{SPF_GT_}\DecValTok{30}\NormalTok{[}\KeywordTok{which}\NormalTok{(data}\OperatorTok{$}\NormalTok{pct_singleparent_hh_}\FloatTok{2017.}\NormalTok{y}\OperatorTok{>=}\FloatTok{0.3}\NormalTok{)] <-}\StringTok{ }\OperatorTok{-}\DecValTok{1} \CommentTok{#471 records}

\NormalTok{data}\OperatorTok{$}\NormalTok{flag_spf <-}\StringTok{ }\KeywordTok{ifelse}\NormalTok{(}\KeywordTok{is.na}\NormalTok{(data}\OperatorTok{$}\NormalTok{SPF_GT_}\DecValTok{30}\NormalTok{), }\DecValTok{0}\NormalTok{, data}\OperatorTok{$}\NormalTok{SPF_GT_}\DecValTok{30}\NormalTok{)}

\KeywordTok{summary}\NormalTok{(data}\OperatorTok{$}\NormalTok{flag_spf) }\CommentTok{#fixed NAs to 0, 471 records}
\end{Highlighting}
\end{Shaded}

\begin{verbatim}
##    Min. 1st Qu.  Median    Mean 3rd Qu.    Max. 
## -1.0000 -1.0000  0.0000 -0.2983  0.0000  0.0000
\end{verbatim}

\subsubsection{2. Filtering Poverty (below 200 FPL) \textgreater{}=
30\%}\label{filtering-poverty-below-200-fpl-30}

returns 418 records with 3 NAs

\begin{Shaded}
\begin{Highlighting}[]
\NormalTok{data}\OperatorTok{$}\NormalTok{POVR200_GT_}\DecValTok{30}\NormalTok{[}\KeywordTok{which}\NormalTok{(data}\OperatorTok{$}\NormalTok{pct_below_200_pov_}\FloatTok{2017.}\NormalTok{x}\OperatorTok{<}\FloatTok{0.3}\NormalTok{)] <-}\StringTok{ }\DecValTok{0}

\NormalTok{data}\OperatorTok{$}\NormalTok{POVR200_GT_}\DecValTok{30}\NormalTok{[}\KeywordTok{which}\NormalTok{(data}\OperatorTok{$}\NormalTok{pct_below_200_pov_}\FloatTok{2017.}\NormalTok{x}\OperatorTok{>=}\FloatTok{0.3}\NormalTok{)] <-}\StringTok{ }\OperatorTok{-}\DecValTok{1} \CommentTok{#418}

\NormalTok{data}\OperatorTok{$}\NormalTok{POVR200_GT_}\DecValTok{30}\NormalTok{ <-}\StringTok{ }\KeywordTok{ifelse}\NormalTok{(}\KeywordTok{is.na}\NormalTok{(data}\OperatorTok{$}\NormalTok{POVR200_GT_}\DecValTok{30}\NormalTok{), }\DecValTok{0}\NormalTok{, data}\OperatorTok{$}\NormalTok{POVR200_GT_}\DecValTok{30}\NormalTok{)}

\KeywordTok{sum}\NormalTok{(data}\OperatorTok{$}\NormalTok{POVR200_GT_}\DecValTok{30}\NormalTok{) }\CommentTok{#418}
\end{Highlighting}
\end{Shaded}

\begin{verbatim}
## [1] -418
\end{verbatim}

\subsubsection{3. Filtering Single parent \textgreater{}= 30\% AND
Poverty (below 200 FPL) \textgreater{}=
30\%}\label{filtering-single-parent-30-and-poverty-below-200-fpl-30}

\begin{Shaded}
\begin{Highlighting}[]
\NormalTok{data}\OperatorTok{$}\NormalTok{SPF30_P30[}\KeywordTok{which}\NormalTok{(data}\OperatorTok{$}\NormalTok{flag_spf}\OperatorTok{==}\DecValTok{0} \OperatorTok{|}\StringTok{ }\NormalTok{data}\OperatorTok{$}\NormalTok{POVR200_GT_}\DecValTok{30}\OperatorTok{==}\DecValTok{0}\NormalTok{)] <-}\StringTok{ }\DecValTok{0}

\NormalTok{data}\OperatorTok{$}\NormalTok{SPF30_P30[}\KeywordTok{which}\NormalTok{(data}\OperatorTok{$}\NormalTok{flag_spf}\OperatorTok{==-}\DecValTok{1} \OperatorTok{&}\StringTok{ }\NormalTok{data}\OperatorTok{$}\NormalTok{POVR200_GT_}\DecValTok{30}\OperatorTok{==-}\DecValTok{1}\NormalTok{)] <-}\StringTok{ }\OperatorTok{-}\DecValTok{1}

\KeywordTok{summary}\NormalTok{(data}\OperatorTok{$}\NormalTok{SPF30_P30) }\CommentTok{#fixed NAs to 0}
\end{Highlighting}
\end{Shaded}

\begin{verbatim}
##    Min. 1st Qu.  Median    Mean 3rd Qu.    Max. 
## -1.0000  0.0000  0.0000 -0.1849  0.0000  0.0000
\end{verbatim}

\begin{Shaded}
\begin{Highlighting}[]
\KeywordTok{sum}\NormalTok{(data}\OperatorTok{$}\NormalTok{SPF30_P30) }\CommentTok{#292}
\end{Highlighting}
\end{Shaded}

\begin{verbatim}
## [1] -292
\end{verbatim}

\subsubsection{4. High Divergence and population of Black and Latinx
\textgreater{}
50\%}\label{high-divergence-and-population-of-black-and-latinx-50}

\begin{Shaded}
\begin{Highlighting}[]
\KeywordTok{load}\NormalTok{(}\KeywordTok{here}\NormalTok{(}\StringTok{"data"}\NormalTok{, }\StringTok{"input_DI.RData"}\NormalTok{))}

\NormalTok{input_DI}\OperatorTok{$}\NormalTok{Flag_HighDI_Blk_Lat[}\KeywordTok{which}\NormalTok{(input_DI}\OperatorTok{$}\NormalTok{Black_Latinx}\OperatorTok{<=}\FloatTok{0.5} \OperatorTok{|}\StringTok{ }\NormalTok{input_DI}\OperatorTok{$}\NormalTok{divergence_thresh}\OperatorTok{<}\DecValTok{3}\NormalTok{)] <-}\StringTok{ }\DecValTok{0}

\NormalTok{input_DI}\OperatorTok{$}\NormalTok{Flag_HighDI_Blk_Lat[}\KeywordTok{which}\NormalTok{(input_DI}\OperatorTok{$}\NormalTok{Black_Latinx}\OperatorTok{>}\FloatTok{0.5} \OperatorTok{&}\StringTok{ }\NormalTok{input_DI}\OperatorTok{$}\NormalTok{divergence_thresh}\OperatorTok{==}\DecValTok{3}\NormalTok{)] <-}\StringTok{ }\OperatorTok{-}\DecValTok{1}

\NormalTok{data}\OperatorTok{$}\NormalTok{fips <-}\StringTok{ }\KeywordTok{paste0}\NormalTok{(}\DecValTok{0}\NormalTok{,data}\OperatorTok{$}\NormalTok{fips)}
\NormalTok{data <-}\StringTok{ }\KeywordTok{merge}\NormalTok{(data,input_DI, }\DataTypeTok{by=}\StringTok{"fips"}\NormalTok{)}
\KeywordTok{sum}\NormalTok{(data}\OperatorTok{$}\NormalTok{Flag_HighDI_Blk_Lat) }\CommentTok{#201 no NAs}
\end{Highlighting}
\end{Shaded}

\begin{verbatim}
## [1] -201
\end{verbatim}

\subsubsection{5. High Divergence with population of Black and Latinx
\textgreater{} 50\% and poverty (below 200 FPL) \textgreater{}=
30\%}\label{high-divergence-with-population-of-black-and-latinx-50-and-poverty-below-200-fpl-30}

\begin{Shaded}
\begin{Highlighting}[]
\NormalTok{data}\OperatorTok{$}\NormalTok{DI_Blk_Lat_POV30[}\KeywordTok{which}\NormalTok{((data}\OperatorTok{$}\NormalTok{Black_Latinx }\OperatorTok{<}\StringTok{ }\FloatTok{0.5} \OperatorTok{&}\StringTok{ }\NormalTok{data}\OperatorTok{$}\NormalTok{divergence_thresh }\OperatorTok{!=}\StringTok{ }\DecValTok{3}\NormalTok{) }\OperatorTok{|}\StringTok{ }\NormalTok{(data}\OperatorTok{$}\NormalTok{POVR200_GT_}\DecValTok{30} \OperatorTok{==}\StringTok{ }\DecValTok{0}\NormalTok{))] <-}\StringTok{ }\DecValTok{0}
\KeywordTok{sum}\NormalTok{(data}\OperatorTok{$}\NormalTok{DI_Blk_Lat_POV30)}
\end{Highlighting}
\end{Shaded}

\begin{verbatim}
## [1] NA
\end{verbatim}

\begin{Shaded}
\begin{Highlighting}[]
\CommentTok{#filter_na <- data %>% select(fips, county_name,DI_Blk_Lat_POV30,Black_Latinx,divergence_thresh,POVR200_GT_30) %>%  filter(is.na(data$DI_Blk_Lat_POV30))}

\NormalTok{data}\OperatorTok{$}\NormalTok{DI_Blk_Lat_POV30[}\KeywordTok{which}\NormalTok{((data}\OperatorTok{$}\NormalTok{Black_Latinx }\OperatorTok{>}\StringTok{ }\FloatTok{0.5} \OperatorTok{&}\StringTok{ }\NormalTok{data}\OperatorTok{$}\NormalTok{divergence_thresh }\OperatorTok{==}\StringTok{ }\DecValTok{3}\NormalTok{) }\OperatorTok{&}\StringTok{ }\NormalTok{(data}\OperatorTok{$}\NormalTok{POVR200_GT_}\DecValTok{30} \OperatorTok{==}\StringTok{ }\OperatorTok{-}\DecValTok{1}\NormalTok{))] <-}\StringTok{ }\OperatorTok{-}\DecValTok{1}
\CommentTok{#320}
\KeywordTok{sum}\NormalTok{(data}\OperatorTok{$}\NormalTok{POVR200_GT_}\DecValTok{30}\NormalTok{) }\CommentTok{#418 no NAs}
\end{Highlighting}
\end{Shaded}

\begin{verbatim}
## [1] -418
\end{verbatim}

\subsubsection{6. Final Filter}\label{final-filter}

High Divergence with population of Black and Latinx \textgreater{} 50\%
and poverty (below 200 FPL) \textgreater{}= 30\% OR Poverty (below 200
FPL) \textgreater{}= 30\% and Single-parent family \textgreater{}= 30\%

\begin{Shaded}
\begin{Highlighting}[]
\NormalTok{data}\OperatorTok{$}\NormalTok{DI_Blk_Lat_POV30_OR_POV30_SPF30[}\KeywordTok{which}\NormalTok{((}
\NormalTok{  data}\OperatorTok{$}\NormalTok{Flag_HighDI_Blk_Lat }\OperatorTok{==}\StringTok{ }\DecValTok{0} \OperatorTok{&}
\StringTok{  }\NormalTok{data}\OperatorTok{$}\NormalTok{POVR200_GT_}\DecValTok{30} \OperatorTok{==}\StringTok{ }\DecValTok{0}
\NormalTok{  ) }\OperatorTok{|}
\StringTok{  }\NormalTok{(}
\NormalTok{  data}\OperatorTok{$}\NormalTok{POVR200_GT_}\DecValTok{30} \OperatorTok{==}\StringTok{ }\DecValTok{0} \OperatorTok{&}
\StringTok{  }\NormalTok{data}\OperatorTok{$}\NormalTok{pct_singleparent_hh_}\FloatTok{2017.}\NormalTok{y }\OperatorTok{<}\StringTok{ }\FloatTok{0.3}
\NormalTok{  )}
\NormalTok{  )] <-}\StringTok{ }\DecValTok{0}

\NormalTok{data}\OperatorTok{$}\NormalTok{DI_Blk_Lat_POV30_OR_POV30_SPF30[}\KeywordTok{which}\NormalTok{((}
\NormalTok{  data}\OperatorTok{$}\NormalTok{Flag_HighDI_Blk_Lat }\OperatorTok{==}\StringTok{ }\OperatorTok{-}\DecValTok{1} \OperatorTok{&}
\StringTok{  }\NormalTok{data}\OperatorTok{$}\NormalTok{POVR200_GT_}\DecValTok{30} \OperatorTok{==}\StringTok{ }\OperatorTok{-}\DecValTok{1}
\NormalTok{  ) }\OperatorTok{|}
\StringTok{  }\NormalTok{(}
\NormalTok{  data}\OperatorTok{$}\NormalTok{POVR200_GT_}\DecValTok{30} \OperatorTok{==}\StringTok{ }\OperatorTok{-}\DecValTok{1} \OperatorTok{&}
\StringTok{  }\NormalTok{data}\OperatorTok{$}\NormalTok{pct_singleparent_hh_}\FloatTok{2017.}\NormalTok{y }\OperatorTok{>=}\StringTok{ }\FloatTok{0.3}
\NormalTok{  )}
\NormalTok{  )] <-}\StringTok{ }\OperatorTok{-}\DecValTok{1}

\NormalTok{data}\OperatorTok{$}\NormalTok{DI_Blk_Lat_POV30_OR_POV30_SPF30 <-}
\StringTok{  }\KeywordTok{ifelse}\NormalTok{(}
  \KeywordTok{is.na}\NormalTok{(data}\OperatorTok{$}\NormalTok{DI_Blk_Lat_POV30_OR_POV30_SPF30),}
  \DecValTok{0}\NormalTok{,}
\NormalTok{  data}\OperatorTok{$}\NormalTok{DI_Blk_Lat_POV30_OR_POV30_SPF30}
\NormalTok{  ) }\CommentTok{#fixed NAs}

\KeywordTok{sum}\NormalTok{(data}\OperatorTok{$}\NormalTok{DI_Blk_Lat_POV30_OR_POV30_SPF30) }\CommentTok{#320 records}
\end{Highlighting}
\end{Shaded}

\begin{verbatim}
## [1] -320
\end{verbatim}

\section{Check filters}\label{check-filters}

\section{Filter Function}\label{filter-function}

\begin{Shaded}
\begin{Highlighting}[]
\CommentTok{#library(dplyr)}
\NormalTok{categorize <-}
\StringTok{  }\NormalTok{data }\OperatorTok\StringTok{ }\NormalTok{dplyr}\OperatorTok{::}\KeywordTok{select}\NormalTok{(}
  \StringTok{"fips"}\NormalTok{,}
  \StringTok{"index"}\NormalTok{,}
  \StringTok{"DI_Blk_Lat_POV30"}\NormalTok{,}
  \StringTok{"SPF30_P30"}\NormalTok{,}
  \StringTok{"DI_Blk_Lat_POV30_OR_POV30_SPF30"}
\NormalTok{  ) }\OperatorTok\StringTok{ }\KeywordTok{filter}\NormalTok{(DI_Blk_Lat_POV30_OR_POV30_SPF30 }\OperatorTok{==}\StringTok{ }\OperatorTok{-}\DecValTok{1}\NormalTok{) }\OperatorTok\StringTok{     }
\StringTok{  }\KeywordTok{mutate}\NormalTok{(}\DataTypeTok{category =}
  \StringTok{"Lowest Opportunity"}\NormalTok{) }\CommentTok{#returns 320}

\NormalTok{remaining <-}
\StringTok{  }\NormalTok{data }\OperatorTok\StringTok{ }\NormalTok{dplyr}\OperatorTok{::}\KeywordTok{select}\NormalTok{(}
  \StringTok{"fips"}\NormalTok{,}
  \StringTok{"index"}\NormalTok{,}
  \StringTok{"DI_Blk_Lat_POV30"}\NormalTok{,}
  \StringTok{"SPF30_P30"}\NormalTok{,}
  \StringTok{"DI_Blk_Lat_POV30_OR_POV30_SPF30"}
\NormalTok{  ) }\OperatorTok\StringTok{ }\KeywordTok{filter}\NormalTok{(}
\NormalTok{  DI_Blk_Lat_POV30_OR_POV30_SPF30 }\OperatorTok{==}\StringTok{ }\DecValTok{0} \OperatorTok{&}
\StringTok{  }\NormalTok{(}
\NormalTok{  fips }\OperatorTok{!=}\StringTok{ "06081984300"} \OperatorTok{&}\StringTok{ }\NormalTok{fips }\OperatorTok{!=}\StringTok{ "06095253000"} \OperatorTok{&}
\StringTok{  }\NormalTok{fips }\OperatorTok{!=}\StringTok{ "06095980000"}
\NormalTok{  )}
\NormalTok{  )}

\NormalTok{remaining}\OperatorTok{$}\NormalTok{DI_Blk_Lat_POV30 <-}\StringTok{ }\DecValTok{0}

\NormalTok{nan_records <-}
\StringTok{  }\NormalTok{data }\OperatorTok\StringTok{ }\NormalTok{dplyr}\OperatorTok{::}\KeywordTok{select}\NormalTok{(}
  \StringTok{"fips"}\NormalTok{,}
  \StringTok{"index"}\NormalTok{,}
  \StringTok{"DI_Blk_Lat_POV30"}\NormalTok{,}
  \StringTok{"SPF30_P30"}\NormalTok{,}
  \StringTok{"DI_Blk_Lat_POV30_OR_POV30_SPF30"}
\NormalTok{  ) }\OperatorTok\StringTok{ }\KeywordTok{filter}\NormalTok{(fips }\OperatorTok{==}\StringTok{ "06081984300"} \OperatorTok{|}
\StringTok{  }\NormalTok{fips }\OperatorTok{==}\StringTok{ "06095253000"} \OperatorTok{|}
\StringTok{  }\NormalTok{fips }\OperatorTok{==}\StringTok{ "06095980000"}\NormalTok{)}

\NormalTok{nan_records}\OperatorTok{$}\NormalTok{category_}\DecValTok{2}\NormalTok{ <-}\StringTok{ }\OtherTok{NA}
\end{Highlighting}
\end{Shaded}

\subsection{D. Categorization with
Filters}\label{d.-categorization-with-filters}

\subsection{E. Categorization without
Filters}\label{e.-categorization-without-filters}

\subsubsection{1. Graphs or Charts}\label{graphs-or-charts}

Plotting index by opportunity categories but not sure why ggplot is not
available

\subsection{F. Missing Values}\label{f.-missing-values}

These are records with NAs or missing values\\
1. fips 06081984300 has NaN in pct\_pov\_below\_200 and
pct\_singleparent\_hh\\
2. fips 06081984300 (Mod) changed to NAs\\
3. fips 06095253000 has NaN in pct\_pov\_below\_200 and
pct\_singleparent\_hh\\
4. fips 06095253000 (Highest) changed to NAs\\
5. fips 06095980000 has NaN in pct\_pov\_below\_200 and
pct\_singleparent\_hh\\
6. fips 06095980000 (High) changed to NAs

\subparagraph{The records below have poverty rate percentages, which is
why they're not changed to NAs to prevent them from not being counted
even if they do not have pct\_single-parent\_ household\_hh. These
records were categorized based on its index
values.}\label{the-records-below-have-poverty-rate-percentages-which-is-why-theyre-not-changed-to-nas-to-prevent-them-from-not-being-counted-even-if-they-do-not-have-pct_single-parent_-household_hh.-these-records-were-categorized-based-on-its-index-values.}

\begin{enumerate}
\def\labelenumi{\arabic{enumi}.}
\setcounter{enumi}{6}
\tightlist
\item
  fips 06001981900 has NA in pct\_singleparent\_hh\\
\item
  fips 06001981900 (High),\\
\item
  fips 06013351101 has NA in pct\_singleparent\_hh\\
\item
  fips 06013351101 (Mod),\\
\item
  fips 06013351102 has NA in pct\_singleparent\_hh\\
\item
  fips 06013351102 (Mod),\\
\item
  fips 06013351103 has NA in pct\_singleparent\_hh\\
\item
  fips 06013351103 (High),\\
\item
  fips 06075980300 has NA in pct\_singleparent\_hh\\
\item
  fips 06075980300 (High)
\end{enumerate}

\subsection{Output: Index with Filters}\label{output-index-with-filters}

\begin{Shaded}
\begin{Highlighting}[]
\KeywordTok{table}\NormalTok{(data}\OperatorTok{$}\NormalTok{category_wo_filters, data}\OperatorTok{$}\NormalTok{category)}
\end{Highlighting}
\end{Shaded}

\begin{verbatim}
## < table of extent 0 x 0 >
\end{verbatim}

\begin{Shaded}
\begin{Highlighting}[]
\NormalTok{merging <-}\StringTok{ }\KeywordTok{full_join}\NormalTok{(nan_records, remaining)}
\end{Highlighting}
\end{Shaded}

\begin{verbatim}
## Joining, by = c("fips", "index", "DI_Blk_Lat_POV30", "SPF30_P30", "DI_Blk_Lat_POV30_OR_POV30_SPF30")
\end{verbatim}

\begin{Shaded}
\begin{Highlighting}[]
\CommentTok{#categories <- full_join(remaining, categorize)}
\CommentTok{#na_records <- index %>% select("fips", "SPF_GT_30" , "pct_singleparent_hh_2017.y") %>% filter(is.na(index$pct_singleparent_hh_2017.y)) #8 NAs}
\CommentTok{#index_w_filters <- full_join(remaining, categorize)}
\CommentTok{#write.csv(index_w_filters, "index_w_filters.csv")}
\end{Highlighting}
\end{Shaded}

\subsubsection{G. Overlays}\label{g.-overlays}

\paragraph{Racial and Ethnic Composition
Overlay}\label{racial-and-ethnic-composition-overlay}

Data Source: ACS Census data 2010-2014\\
Description: To analyze the distribution of racial and ethnic
composition. I joined the shapefile using the `GEOID' field to match it
with the GEOID in the opportunity categories shapefile

\begin{Shaded}
\begin{Highlighting}[]
\NormalTok{race_list <-}\StringTok{ }\KeywordTok{c}\NormalTok{(}\StringTok{"fips"}\NormalTok{, }\StringTok{"CountyID.x.x"}\NormalTok{, }\StringTok{"TOTPOP.x.x"}\NormalTok{, }\StringTok{"total_pop"}\NormalTok{,}\StringTok{"index"}\NormalTok{,}\StringTok{"total_pop"}\NormalTok{,}\StringTok{"white"}\NormalTok{,}\StringTok{"black"}\NormalTok{, }\StringTok{"asian"}\NormalTok{, }\StringTok{"hispanic"}\NormalTok{, }\StringTok{"other"}\NormalTok{, }\StringTok{"county_name"}\NormalTok{,}\StringTok{"cbsa"}\NormalTok{,}\StringTok{"divergence_thresh"}\NormalTok{, }\StringTok{"DI_Blk_Lat_POV30_OR_POV30_SPF30"}\NormalTok{)}

\CommentTok{#filter_race <- function(data, r_list)\{}
\CommentTok{#  daf <- data %>% select(r_list)}
\CommentTok{#  assign(daf, envir=.GlobalEnv)}
\CommentTok{#\}}
\CommentTok{#source(here(filter_race.R))}
\CommentTok{#filter_race(data=index_filters, r_list=race_list)}
\end{Highlighting}
\end{Shaded}

\paragraph{Median Household Income}\label{median-household-income}

Data Source: American Community Survey (5-year-estimates)\\
Table: B19013\_001 -- MEDIAN HOUSEHOLD INCOME IN THE PAST 12 MONTHS (IN
2017 INFLATION-ADJUSTED DOLLARS)

\paragraph{Payday Lending Overlay}\label{payday-lending-overlay}

Data Source: ESRI Business Analyst\\
Spreadsheet: OV\_YEAR\_Payday\\
Description: 2017 Measure -- Spatially join the payday lending in the
bay area shape file to the 2014 census tract shape file with the
opportunity categories to obtain the number of businesses per census
tract. Then use the count of number of businesses per tract divided by
the total count number of payday lending and credit businesses in the
Bay Area to obtain the percentage.\\
2018 Measure -- Identify whether the column salevolume in the dataset
has the volume of payday loan sales. Aggregate those sales and
distribute them to tracts to identify the amount of sales in each
neighborhood OR (if it's possible to) identity where the highest
percentage of interests (200-400\%) that these payday loans are located
and how many of them are in each census tracts.

\begin{Shaded}
\begin{Highlighting}[]
\CommentTok{#load(file="BA_payday_2018.RData")}
\CommentTok{#proj4string(BA_payday_2018)}
\end{Highlighting}
\end{Shaded}

\paragraph{Subsidized Housing Overlay}\label{subsidized-housing-overlay}

Data Source: HUD subsidized housing projects\\
Spreadsheet: OV\_Year\_SubHous\\
Description:

• Data should be gathered through HUD instead of TCAC. Use the file
obtained from HUD to create a point shapefile based on the lat and long
for each (which is in the table).\\
• This table has all subsidized housing projects in California; Use
geoprocessing to clip the subsidized housing shapefile to Bay Area\\
• Analysis of Projects and Units should be included in the map based on
subsidized units available and the number of subsidized programs in the
region.

\paragraph{Low population density
Overlay}\label{low-population-density-overlay}

Data Source: Census Data\\
Spreadsheet: OV\_Year\_LowDen\\
Description: To analyze the density of the census tract and identify
areas that are considered low density with 40 or more acres per person\\
• Calculate the ``area'' of each tract in acres. Then I divided that by
the number of people, and the results are in POP\_DEN field. All tracts
which had a value of 40 or above were highlighted on the map with a
specific symbology\\
Example:\\
Step 1: Create a new field, ``Acres\_per'' person for each tract
\textgreater{} Calculate Geometry \textgreater{} selecting Area
\textgreater{} Coordinate System: Use Coordinate System of the data
frame: PCS: NAD 1983 StatePlane California III FIPS 0403 \textgreater{}
Units: Acres {[}US{]} (ac) \textgreater{} OK\\
Step 2: Then, create a new field titled, ``POP\_DEN'' in which the value
would be ``Acres\_per'' person for each tract divided by the number of
people in the tract \textgreater{} select the tracts that have the value
of 40 or above


\end{document}
