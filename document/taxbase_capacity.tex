\documentclass[]{article}
\usepackage{lmodern}
\usepackage{amssymb,amsmath}
\usepackage{ifxetex,ifluatex}
\usepackage{fixltx2e} % provides \textsubscript
\ifnum 0\ifxetex 1\fi\ifluatex 1\fi=0 % if pdftex
  \usepackage[T1]{fontenc}
  \usepackage[utf8]{inputenc}
\else % if luatex or xelatex
  \ifxetex
    \usepackage{mathspec}
  \else
    \usepackage{fontspec}
  \fi
  \defaultfontfeatures{Ligatures=TeX,Scale=MatchLowercase}
\fi
% use upquote if available, for straight quotes in verbatim environments
\IfFileExists{upquote.sty}{\usepackage{upquote}}{}
% use microtype if available
\IfFileExists{microtype.sty}{%
\usepackage{microtype}
\UseMicrotypeSet[protrusion]{basicmath} % disable protrusion for tt fonts
}{}
\usepackage[margin=1in]{geometry}
\usepackage{hyperref}
\hypersetup{unicode=true,
            pdftitle={School District Revenue Per Capita},
            pdfauthor={Phuong Tseng},
            pdfborder={0 0 0},
            breaklinks=true}
\urlstyle{same}  % don't use monospace font for urls
\usepackage{color}
\usepackage{fancyvrb}
\newcommand{\VerbBar}{|}
\newcommand{\VERB}{\Verb[commandchars=\\\{\}]}
\DefineVerbatimEnvironment{Highlighting}{Verbatim}{commandchars=\\\{\}}
% Add ',fontsize=\small' for more characters per line
\usepackage{framed}
\definecolor{shadecolor}{RGB}{248,248,248}
\newenvironment{Shaded}{\begin{snugshade}}{\end{snugshade}}
\newcommand{\KeywordTok}[1]{\textcolor[rgb]{0.13,0.29,0.53}{\textbf{#1}}}
\newcommand{\DataTypeTok}[1]{\textcolor[rgb]{0.13,0.29,0.53}{#1}}
\newcommand{\DecValTok}[1]{\textcolor[rgb]{0.00,0.00,0.81}{#1}}
\newcommand{\BaseNTok}[1]{\textcolor[rgb]{0.00,0.00,0.81}{#1}}
\newcommand{\FloatTok}[1]{\textcolor[rgb]{0.00,0.00,0.81}{#1}}
\newcommand{\ConstantTok}[1]{\textcolor[rgb]{0.00,0.00,0.00}{#1}}
\newcommand{\CharTok}[1]{\textcolor[rgb]{0.31,0.60,0.02}{#1}}
\newcommand{\SpecialCharTok}[1]{\textcolor[rgb]{0.00,0.00,0.00}{#1}}
\newcommand{\StringTok}[1]{\textcolor[rgb]{0.31,0.60,0.02}{#1}}
\newcommand{\VerbatimStringTok}[1]{\textcolor[rgb]{0.31,0.60,0.02}{#1}}
\newcommand{\SpecialStringTok}[1]{\textcolor[rgb]{0.31,0.60,0.02}{#1}}
\newcommand{\ImportTok}[1]{#1}
\newcommand{\CommentTok}[1]{\textcolor[rgb]{0.56,0.35,0.01}{\textit{#1}}}
\newcommand{\DocumentationTok}[1]{\textcolor[rgb]{0.56,0.35,0.01}{\textbf{\textit{#1}}}}
\newcommand{\AnnotationTok}[1]{\textcolor[rgb]{0.56,0.35,0.01}{\textbf{\textit{#1}}}}
\newcommand{\CommentVarTok}[1]{\textcolor[rgb]{0.56,0.35,0.01}{\textbf{\textit{#1}}}}
\newcommand{\OtherTok}[1]{\textcolor[rgb]{0.56,0.35,0.01}{#1}}
\newcommand{\FunctionTok}[1]{\textcolor[rgb]{0.00,0.00,0.00}{#1}}
\newcommand{\VariableTok}[1]{\textcolor[rgb]{0.00,0.00,0.00}{#1}}
\newcommand{\ControlFlowTok}[1]{\textcolor[rgb]{0.13,0.29,0.53}{\textbf{#1}}}
\newcommand{\OperatorTok}[1]{\textcolor[rgb]{0.81,0.36,0.00}{\textbf{#1}}}
\newcommand{\BuiltInTok}[1]{#1}
\newcommand{\ExtensionTok}[1]{#1}
\newcommand{\PreprocessorTok}[1]{\textcolor[rgb]{0.56,0.35,0.01}{\textit{#1}}}
\newcommand{\AttributeTok}[1]{\textcolor[rgb]{0.77,0.63,0.00}{#1}}
\newcommand{\RegionMarkerTok}[1]{#1}
\newcommand{\InformationTok}[1]{\textcolor[rgb]{0.56,0.35,0.01}{\textbf{\textit{#1}}}}
\newcommand{\WarningTok}[1]{\textcolor[rgb]{0.56,0.35,0.01}{\textbf{\textit{#1}}}}
\newcommand{\AlertTok}[1]{\textcolor[rgb]{0.94,0.16,0.16}{#1}}
\newcommand{\ErrorTok}[1]{\textcolor[rgb]{0.64,0.00,0.00}{\textbf{#1}}}
\newcommand{\NormalTok}[1]{#1}
\usepackage{graphicx,grffile}
\makeatletter
\def\maxwidth{\ifdim\Gin@nat@width>\linewidth\linewidth\else\Gin@nat@width\fi}
\def\maxheight{\ifdim\Gin@nat@height>\textheight\textheight\else\Gin@nat@height\fi}
\makeatother
% Scale images if necessary, so that they will not overflow the page
% margins by default, and it is still possible to overwrite the defaults
% using explicit options in \includegraphics[width, height, ...]{}
\setkeys{Gin}{width=\maxwidth,height=\maxheight,keepaspectratio}
\IfFileExists{parskip.sty}{%
\usepackage{parskip}
}{% else
\setlength{\parindent}{0pt}
\setlength{\parskip}{6pt plus 2pt minus 1pt}
}
\setlength{\emergencystretch}{3em}  % prevent overfull lines
\providecommand{\tightlist}{%
  \setlength{\itemsep}{0pt}\setlength{\parskip}{0pt}}
\setcounter{secnumdepth}{0}
% Redefines (sub)paragraphs to behave more like sections
\ifx\paragraph\undefined\else
\let\oldparagraph\paragraph
\renewcommand{\paragraph}[1]{\oldparagraph{#1}\mbox{}}
\fi
\ifx\subparagraph\undefined\else
\let\oldsubparagraph\subparagraph
\renewcommand{\subparagraph}[1]{\oldsubparagraph{#1}\mbox{}}
\fi

%%% Use protect on footnotes to avoid problems with footnotes in titles
\let\rmarkdownfootnote\footnote%
\def\footnote{\protect\rmarkdownfootnote}

%%% Change title format to be more compact
\usepackage{titling}

% Create subtitle command for use in maketitle
\providecommand{\subtitle}[1]{
  \posttitle{
    \begin{center}\large#1\end{center}
    }
}

\setlength{\droptitle}{-2em}

  \title{School District Revenue Per Capita}
    \pretitle{\vspace{\droptitle}\centering\huge}
  \posttitle{\par}
    \author{Phuong Tseng}
    \preauthor{\centering\large\emph}
  \postauthor{\par}
      \predate{\centering\large\emph}
  \postdate{\par}
    \date{May 2019}


\begin{document}
\maketitle

{
\setcounter{tocdepth}{2}
\tableofcontents
}
\subsection{Information about the
Script:}\label{information-about-the-script}

\begin{quote}
This script contains procedures and explanations for cleaning and
calculating the school district revenue per capita indicator. There are
4 important steps and 3 different datasets in this analysis. The first
step is to clean and filter the finance data. Then, clean the Government
Integrated Directory (GID) dataset, which has the jurisdictions, and
remove all cities that are not within the bay area. The third step
merges both finance and GID with the geographic relationship file (GRF).
\end{quote}

\begin{quote}
\begin{quote}
{[}For the survey's methodology{]}
(\url{https://www2.census.gov/programs-surveys/gov-finances/technical-documentation/methodology/2016/2016\%20Methodology\%20Document\%20Final.pdf}.)
Contact Information: 1-800-832-2839. Data Source: 2016 Census of
Governments Survey of Local Government Finances. Credit to Heather
Bromfield for developing the initial analysis in March 2018 in Stata.
Reviewer and editor: Arthur Gailes in November 2018. Phuong Tseng from
May 2018 to May 2019.
\end{quote}
\end{quote}

\subsection{1. Setup the packages and
libraries}\label{setup-the-packages-and-libraries}

To work with this Rmarkdown file, there are a few packages such as
\textbf{dplyr, data.table, readr, readxl, stringr, and RCurl} that you
need to run.

\subsubsection{A. Read in the dataset}\label{a.-read-in-the-dataset}

Next, read in the finance survey dataset from the Census of Governments.
The dataset does not have a header but it has two columns titled, ``X1''
and ``X2'' separated by spaces.

\begin{verbatim}
## [1] "X1" "X2"
\end{verbatim}

\subsubsection{B. Extract and rename the column
names}\label{b.-extract-and-rename-the-column-names}

To work with this dataset, use \textbf{stringr::substr} to extract the
position and length of each vector.

\begin{verbatim}
## [1] "X1"       "X2"       "state"    "gov_type" "cnty"     "flag"    
## [7] "unit"     "own_unit" "taxcode"
\end{verbatim}

\subsubsection{C. Filter the dataset}\label{c.-filter-the-dataset}

After you extracted and renamed the columns, you now have to filter the
dataset by tax codes and government type within California.

\begin{Shaded}
\begin{Highlighting}[]
\NormalTok{schooldistrict <-}
\StringTok{  }\KeywordTok{filter}\NormalTok{(X2016_FinEst,}
         \KeywordTok{grepl}\NormalTok{(}\StringTok{'^[AD-DT-U]'}\NormalTok{, taxcode),}
\NormalTok{         state }\OperatorTok{==}\StringTok{ "05"}\NormalTok{,}
\NormalTok{         gov_type }\OperatorTok{==}\StringTok{ "5"}\NormalTok{) }\OperatorTok\StringTok{ }\KeywordTok{select}\NormalTok{(X1, X2, state, cnty, gov_type, flag, unit, }
\NormalTok{                                     own_unit, taxcode, flag)}
  
  \KeywordTok{unique}\NormalTok{(schooldistrict}\OperatorTok{$}\NormalTok{taxcode) }\CommentTok{#Return tax codes with A,D,T, and U.}
\end{Highlighting}
\end{Shaded}

\begin{verbatim}
##  [1] "A09" "A10" "A12" "D11" "D21" "T01" "T99" "U20" "U99" "U11" "A18"
## [12] "U50" "A16"
\end{verbatim}

\begin{Shaded}
\begin{Highlighting}[]
  \KeywordTok{dim}\NormalTok{(schooldistrict) }\CommentTok{#This should return 6,608 school district records}
\end{Highlighting}
\end{Shaded}

\begin{verbatim}
## [1] 6608    9
\end{verbatim}

\subsubsection{D. Extract the school district revenue
amount}\label{d.-extract-the-school-district-revenue-amount}

There are about 31,506 records within CA but there are no record of
townships. There are 4,110 special districts and 18,225 school districts
in the US. Once the revenue is extracted from the \textbf{``X2'' and
``test\_amount''} columns, the revenue amount needs to multiply it by
1000 because these amounts are in thousands.

\begin{Shaded}
\begin{Highlighting}[]
\NormalTok{schooldistrict}\OperatorTok{$}\NormalTok{test_amount <-}
\StringTok{  }\KeywordTok{str_trim}\NormalTok{(schooldistrict}\OperatorTok{$}\NormalTok{X2, }\DataTypeTok{side =} \StringTok{"both"}\NormalTok{)}
\NormalTok{  schooldistrict}\OperatorTok{$}\NormalTok{amount <-}
\StringTok{  }\KeywordTok{str_sub}\NormalTok{(schooldistrict}\OperatorTok{$}\NormalTok{test_amount, }\DataTypeTok{end =} \OperatorTok{-}\DecValTok{6}\NormalTok{)}
\NormalTok{  schooldistrict <-}
\StringTok{  }\NormalTok{schooldistrict }\OperatorTok\StringTok{ }\KeywordTok{mutate}\NormalTok{(}\DataTypeTok{amount =} \KeywordTok{as.numeric}\NormalTok{(amount), }\DataTypeTok{amount_2016 =}\NormalTok{ amount }\OperatorTok{*}
\StringTok{  }\DecValTok{1000}\NormalTok{)}
  \KeywordTok{head}\NormalTok{(schooldistrict}\OperatorTok{$}\NormalTok{amount_}\DecValTok{2016}\NormalTok{)}
\end{Highlighting}
\end{Shaded}

\begin{verbatim}
## [1]   672000     5000   269000  5429000  1645000 38367000
\end{verbatim}

\subsubsection{E. Generate a new ID}\label{e.-generate-a-new-id}

Use columns \textbf{state, gov\_type, cnty, unit, and own\_unit} to
generate a 14-digits ID to be used in part 2

\subsection{2. Clean the Government Integrated Directory
file}\label{clean-the-government-integrated-directory-file}

\begin{verbatim}
##  [1] "X__1"        "X__2"        "X__3"        "X__4"        "X__5"       
##  [6] "X__6"        "state"       "jrsd_type"   "cnty"        "etc"        
## [11] "jrdsct_name" "statecode"   "countycode"
\end{verbatim}

\subsubsection{A. Extract GID records in bay
area}\label{a.-extract-gid-records-in-bay-area}

\begin{Shaded}
\begin{Highlighting}[]
\NormalTok{Fin_GID_}\DecValTok{2016}\NormalTok{ <-}
\StringTok{  }\NormalTok{dplyr}\OperatorTok{::}\KeywordTok{filter}\NormalTok{(}
\NormalTok{  Fin_GID_2016_copy,}
\NormalTok{  X__}\DecValTok{3} \OperatorTok{==}\StringTok{ "Alameda"} \OperatorTok{|}
\StringTok{  }\NormalTok{X__}\DecValTok{3} \OperatorTok{==}\StringTok{ "Contra Costa"} \OperatorTok{|}
\StringTok{  }\NormalTok{X__}\DecValTok{3} \OperatorTok{==}\StringTok{ "Marin"} \OperatorTok{|}
\StringTok{  }\NormalTok{X__}\DecValTok{3} \OperatorTok{==}\StringTok{ "Napa"} \OperatorTok{|}
\StringTok{  }\NormalTok{X__}\DecValTok{3} \OperatorTok{==}\StringTok{ "Santa Clara"} \OperatorTok{|}
\StringTok{  }\NormalTok{X__}\DecValTok{3} \OperatorTok{==}\StringTok{ "San Mateo"} \OperatorTok{|}
\StringTok{  }\NormalTok{X__}\DecValTok{3} \OperatorTok{==}\StringTok{ "San Francisco"} \OperatorTok{|}\StringTok{ }\NormalTok{X__}\DecValTok{3} \OperatorTok{==}\StringTok{ "Solano"} \OperatorTok{|}\StringTok{ }\NormalTok{X__}\DecValTok{3} \OperatorTok{==}\StringTok{ "Sonoma"}
\NormalTok{  )}

\KeywordTok{dim}\NormalTok{(Fin_GID_}\DecValTok{2016}\NormalTok{)}
\end{Highlighting}
\end{Shaded}

\begin{verbatim}
## [1] 273  13
\end{verbatim}

This returns 273 records.

\subsubsection{B. Generate a new ID}\label{b.-generate-a-new-id}

Use columns \textbf{state, gov\_type, cnty, unit, and own\_unit} to
generate a 14-digits ID to be used in part 2

\begin{Shaded}
\begin{Highlighting}[]
\CommentTok{#Fin_GID_2016$ID <- paste0(Fin_GID_2016$state, Fin_GID_2016$jrsd_type, Fin_GID_2016$cnty, Fin_GID_2016$etc) }
\end{Highlighting}
\end{Shaded}

This sample has all 9 counties

\subsubsection{Merge the GID and Finance
file}\label{merge-the-gid-and-finance-file}

\begin{Shaded}
\begin{Highlighting}[]
\CommentTok{#merge_districts <- merge(Fin_GID_2016, schooldistrict, by="ID") #1247 school districts with all 9 counties}
\CommentTok{#merge_districts_total <- merge_districts %>% group_by(ID, X__1, X__2, X__3, X__4, jrsd_type,cnty.x, jrdsct_name, statecode, countycode, flag,unit, own_unit,amount,amount_2016) %>% mutate(district_amount_2016=sum(amount_2016)) #NEED TO CHECK}

\CommentTok{#summary(merge_districts_total)}
\end{Highlighting}
\end{Shaded}

\subsubsection{Read the school district population
file}\label{read-the-school-district-population-file}

\subsubsection{Read the geographic relationship file and merge it with
the population
file}\label{read-the-geographic-relationship-file-and-merge-it-with-the-population-file}

\subsubsection{Fix jurisdiction names}\label{fix-jurisdiction-names}

An alternative to fixing the jurisdiction names is to match the school
district IDs to its school names.

\subsection{Export the files}\label{export-the-files}

\subsection{Output code to script}\label{output-code-to-script}


\end{document}
